\documentclass[a4paper,12pt]{article}
\usepackage[cp1250]{inputenc}
\usepackage{here}
\usepackage{Sweave}
\usepackage{natbib}
\bibpunct{(}{)}{;}{a}{,}{,}
% \usepackage[nogin]{Sweave} % keep R figure size
    %%%%%%%%%%%%%%%%%%%%%%%%%%%%%%
    \oddsidemargin  0.0in
    \evensidemargin 0.0in
    \textwidth      6.0in
    \headheight     0.0in
    \topmargin      -0.75in
    \textheight=    10.0in
    %%%%%%%%%%%%%%%%%%%%%%%%%%%%%%
% Alter some LaTeX defaults for better treatment of figures:
    % See p.105 of "TeX Unbound" for suggested values.
    % See pp. 199-200 of Lamport's "LaTeX" book for details.
    % http://mintaka.sdsu.edu/GF/bibliog/latex/floats.html
    %   General parameters, for ALL pages:
    \renewcommand{\topfraction}{0.9}    % max fraction of floats at top
    \renewcommand{\bottomfraction}{0.8} % max fraction of floats at bottom
    %   Parameters for TEXT pages (not float pages):
    \setcounter{topnumber}{2}
    \setcounter{bottomnumber}{2}
    \setcounter{totalnumber}{4}     % 2 may work better
    \setcounter{dbltopnumber}{2}    % for 2-column pages
    \renewcommand{\dbltopfraction}{0.9} % fit big float above 2-col. text
    \renewcommand{\textfraction}{0.07}  % allow minimal text w. figs
    %   Parameters for FLOAT pages (not text pages):
    \renewcommand{\floatpagefraction}{0.7}  % require fuller float pages
    % N.B.: floatpagefraction MUST be less than topfraction !!
    \renewcommand{\dblfloatpagefraction}{0.7}   % require fuller float pages

    % remember to use [htp] or [htpb] for placement

%\newcommand{\R}{\textsf{R }}
\newcommand{\R}{{\normalfont\textsf{R}}{}}
\newcommand{\degC}{$^\circ$C} %\hspace{8pt}}
\newcommand{\degCs}{$^\circ$C\hspace{8pt}}
%\newtheorem{zakljucek}{\textbf{Zaklju�ek}}[section]
\newcounter{zakljucki}[section]
\newtheorem{naloga}{Naloga}
\newenvironment{Naloga}
    {\begin{naloga}\rm}{\end{naloga}}
\newenvironment{conclusion}[1][Zaklju�ek]
{\textbf{\\ \bigskip \stepcounter{zakljucki} \hrule \bigskip \noindent #1
\arabic{section}.\arabic{zakljucki}} \par} {\bigskip \hrule
\bigskip}

\citeindextrue

\title{Statistika\\Vaja 1}
\author{Andrej Blejec}

\begin{document}
\maketitle
\section{Urejanje podatkov}

\begin{Naloga}
Za vsako od na�tetih merskih lestvic dolo�ite mo�ne vrednosti in jo opredelite glede na
vsebino, zna�aj in zveznost. Poskusite dolociti tudi zna�aj pojava, ki ga merimo.

\begin{enumerate}
    \renewcommand{\theenumi}{\alph{enumi}}

    \item �tevilo otrok v dru�ini
    \item �as od vsaditve semena do kalitve
    \item Reakcije �ivali, ozna�ene od 1 do 5:
    \\1 zelo agresivna, 2 agresivna, 3 nevtralna,  4 prestra�ena, 5 zelo prestra�ena
    \item �tevilo rastlin v poskusnem kvadratu
    \item pH
    \item dol�ina mi�jega repa
    \item suha te�a planktona iz litra vode
    \item temperatura zraka, merjena vsake pol ure na stopinjo natan�no
    \item skupno �tevilo pik pri metu treh kock
    \item pozivna telefonska �tevilka kraja
    \item vrsta domace �ivali: 1 pes, 2 macka, 3 papiga, 9 drugo
    \item �as, ki ga potrebuje podgana da pride iz labirinta
    \item sprememba telesne te�e v 6 mesecih (merjena v kg)
    \item razlika v temperaturi po in pred poskusom (v stopinjah K)
    \item �irina glave merjena kot: 3 �iroka 2 srednja, 1 ozka
    \item datum rojstva
    \renewcommand{\theenumi}{\arabic{enumi}}

\end{enumerate}
\end{Naloga}


\end{document}
